%%%%%%%%%%%%%Manuscript tutorial by C Has%%%%%%%%%%%%%%%%%%%%%%%%%%%%%%%%%%%%%
\documentclass[journal=jacsat,manuscript=article]{achemso}

\usepackage{mathptmx}

\title{Manuscript Writing for ACS in \LaTeX}

\author{Author A}
\author{Author B}
\affiliation{Common address for Author A and Author B}

\author{Author C}
\email{author-c@email.com}
%\phone{+91(0)22 25764237}
%\fax{+91(0)22 25764237}
\altaffiliation{Address for Author C}
\keywords{American Chemical Society, \LaTeX}

\begin{document}

\begin{abstract}
  This is an example document for the \textsf{achemso} document
  class, intended for submissions to the American Chemical Society
  for publication. The class is based on the standard \LaTeXe\
  \textsf{report} file, and does not seek to reproduce the appearance
  of a published paper.  This is an abstract for the \textsf{achemso} document class
  demonstration document.  An abstract is only allowed for certain
  manuscript types.  The selection of \texttt{journal} and
  \texttt{manuscript} will determine if an abstract is valid.  If
  not, the class will issue an appropriate error.
\end{abstract}

\section{Introduction}
This is a paragraph of text to fill the introduction of the
demonstration file.  The demonstration file attempts to show the
modifications of the standard \LaTeX\ macros that are implemented by
the \textsf{achemso} class.  These are mainly concerned with content,
as opposed to appearance.

\section{Results and discussion}

\subsection{Outline}
The document layout should follow the style of the journal concerned.
Where appropriate, sections and subsections should be added in the
normal way. If the class options are set correctly, warnings will be
given if these should not be present.

\subsection{References}
The class makes various changes to the way that references are
handled.  The class loads \textsf{natbib}, and also the
appropriate bibliography style.  References can be made using
the normal method; the citation should be placed before any
punctuation, as the class will move it if using a superscript
citation style
\cite{Mena2000,Abernethy2003,Friedman-Hill2003,EuropeanCommission2008}.
The use of \textsf{natbib} allows the use of the various citation
commands of that package: \citeauthor{Abernethy2003} have shown
something, in \citeyear{Cotton1999}, or as given by
Ref.~\citenum{Mena2000}.  Long lists of authors will be
automatically truncated in most article formats, but not in
supplementary information or reviews \cite{Pople2003}. If you
encounter problems with the citation macros, please check that
your copy of \textsf{natbib} is up to date. The demonstration
database file \texttt{achemso-demo.bib} shows how to complete
entries correctly. Notice that ``\latin{et al.}'' is auto-formatted
using the \texttt{\textbackslash latin} command.

Multiple citations to be combined into a list can be given as
a single citation.  This uses the \textsf{mciteplus} package
\cite{Johnson1972,*Arduengo1992,*Eisenstein2005,*Arduengo1994}.
Citations other than the first of the list should be indicated
with a star. If the \textsf{mciteplus} package is not installed,
the standard bibliography tools will still work but starred
references will be ignored. Individual references can be referred
to using \texttt{\textbackslash mciteSubRef}:
``ref.~\mciteSubRef{Eisenstein2005}''.

The class also handles notes to be added to the bibliography.  These
should be given in place in the document \bibnote{This is a note.
The text will be moved the the references section.  The title of the
section will change to ``Notes and References''.}.  As with
citations, the text should be placed before punctuation.  A note is
also generated if a citation has an optional note.  This assumes that
the whole work has already been cited: odd numbering will result if
this is not the case \cite[p.~1]{Cotton1999}.

\subsection{Floats}
New float types are automatically set up by the class file.  The
means graphics are included as follows (Scheme~\ref{sch:example}).  As
illustrated, the float is ``here'' if possible.
\begin{scheme}
  Your scheme graphic would go here: \texttt{.eps} format\\
  for \LaTeX\, or \texttt{.pdf} (or \texttt{.png}) for pdf\LaTeX\\
  \textsc{ChemDraw} files are best saved as \texttt{.eps} files:\\
  these can be scaled without loss of quality, and can be\\
  converted to \texttt{.pdf} files easily using \texttt{eps2pdf}.\\
  %\includegraphics{graphic}
  \caption{An example scheme}
  \label{sch:example}
\end{scheme}

\begin{figure}[tbp]
	\centering
	\includegraphics[width=0.35\linewidth]{gfx}
	\caption{Lorem Ipsum is simply dummy text of the printing and typesetting industry.}
	\label{fig:logo}
\end{figure}

\subsection{Math(s)}
The \textsf{achemso} class does not load any particular additional
support for mathematics.  If packages such as \textsf{amsmath} are
required, they should be loaded in the preamble.  However,
the basic \LaTeX\ math(s) input should work correctly without
this.  Some inline material \( y = mx + c \) or $ 1 + 1 = 2 $
followed by some display. \[ A = \pi r^2 \]

It is possible to label equations in the usual way (Eq.~\ref{eqn:example}).
\begin{equation}
  \frac{\mathrm{d}}{\mathrm{d}x} \, r^2 = 2r \label{eqn:example}
\end{equation}

\section{Experimental}
The usual experimental details should appear here.  This could
include a table, which can be referenced as Table~\ref{tbl:example}.
Notice that the caption is positioned at the top of the table.
\begin{table}
  \caption{An example table}
  \label{tbl:example}
  \begin{tabular}{ll}
    \hline
    Header one  & Header two  \\
    \hline
    Entry one   & Entry two   \\
    Entry three & Entry four  \\
    Entry five  & Entry five  \\
    Entry seven & Entry eight \\
    \hline
  \end{tabular}
\end{table}

\begin{acknowledgement}
Please use ``The authors thank \ldots'' rather than ``The
authors would like to thank \ldots''.

The author thanks Mats Dahlgren for version one of \textsf{achemso},
and Donald Arseneau for the code taken from \textsf{cite} to move
citations after punctuation. Many users have provided feedback on the
class, which is reflected in all of the different demonstrations
shown in this document.
\end{acknowledgement}

\begin{suppinfo}
A listing of the contents of each file supplied as Supporting Information
should be included. For instructions on what should be included in the
Supporting Information as well as how to prepare this material for
publications, refer to the journal's Instructions for Authors.

The following files are available free of charge.
\begin{itemize}
  \item Filename: brief description
  \item Filename: brief description
\end{itemize}
\end{suppinfo}

\bibliography{achemso-bib}

\end{document}
